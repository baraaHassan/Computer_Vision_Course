\documentclass[11pt]{article}

\usepackage{amsmath}
\usepackage{amsfonts}
\usepackage{url}
\usepackage{graphicx}
\usepackage[a4paper,top=1.2in, bottom=1.2in, left=1.2in, right=1.2in]{geometry}
\usepackage{ifthen}

% no indent
\setlength\parindent{0pt}

% points
\newcommand{\points}[1]{
    \ifthenelse{\equal{#1}{1}}
        {\\ \emph{(#1 Point)}}
    % else
        {\\ \emph{(#1 Points)}}
}

\begin{document}

\begin{center}
\textbf{Exercise 9 for MA-INF 2201 Computer Vision WS17/18\\
18.12.2017\\
Submission on 07.01.2018\\
Optical Flow}\\
\end{center}

\vspace{1cm}

\begin{enumerate}
    % exercise 1
    \item \textbf{Lucas-Kanade Flow}:

        You can find two consecutive frames and the optical flow ground truth in the \texttt{data} directory. We provide two functions that may help you solving the following tasks.
        The function \texttt{read\_FLO\_file} can be used to read the FLO-file containing the ground truth. To convert the optical flow to an RGB image, you can use the function \texttt{flow\_to\_RGB}.
        \begin{enumerate}
            \item Implement your own version of the Lucas-Kanade optical flow as presented in the lecture. Use a $ 15 \times 15 $ window in the algorithm. Display the ground truth and your estimated flow.
            \points{9}
            \item Report the average angular error of your estimated flow.
            \points{1}
        \end{enumerate}
    \item \textbf{Horn-Schunck Flow}:

    Implement your own version of the Horn-Schunck optical flow using an iterative optimization based on the Jacobi method as originally proposed by Horn and Schunck\footnote{B.K.P. Horn and B.G. Schunck, \textit{Determining optical flow}. Artificial Intelligence, vol. 17, pp. 185 -- 203, 1981}.
    The iterative update rule is defined by
    \begin{align}
        u^{(k+1)} &= \bar u^{(k)} - \frac{I_x(I_x \bar u^{(k)} + I_y \bar v^{(k)} + I_t)}{\alpha^2 + I_x^2 + I_y^2}, \\
        v^{(k+1)} &= \bar v^{(k)} - \frac{I_y(I_x \bar u^{(k)} + I_y \bar v^{(k)} + I_t)}{\alpha^2 + I_x^2 + I_y^2},
    \end{align}
    where
    \begin{align}
        \bar u^{(k)} = u^{(k)} + \Delta u^{(k)} \quad \text{and} \quad
        \bar v^{(k)} = v^{(k)} + \Delta v^{(k)}.
    \end{align}
    You can approximate the laplacian $ \Delta u^{(k)} $ and $ \Delta v^{(k)} $ using the normalized laplacian kernel
    \begin{align}
        K = \begin{pmatrix} 0 & \frac{1}{4} & 0 \\ \frac{1}{4} & -1 & \frac{1}{4} \\ 0 & \frac{1}{4} & 0 \end{pmatrix}.
    \end{align}

    Use your implementation to estimate the optical flow on the two given frames. Set $ \alpha = 1 $ and initialize
    $ u^{(0)} $ and $ v^{(0)} $ with zero. Iterate until the difference of two flow fields in $ L_1 $ norm is less than
    $ 0.002 $, i.e.\ until
    \begin{align}
        \sum_{i,j} |u^{(k+1)}_{i,j} - u^{(k)}_{i,j}| + |v^{(k+1)}_{i,j} - v^{(k)}_{i,j}| < 0.002.
    \end{align}
    Report the average angular error and display the estimated flow.
    \points{10}
\end{enumerate}

\end{document}
